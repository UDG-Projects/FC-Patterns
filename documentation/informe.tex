\documentclass[12pt,a4paper]{report}
\usepackage{titlesec}
\usepackage{graphicx}
\usepackage[utf8]{inputenc}
\usepackage[dvipsnames]{xcolor}
\usepackage{color}
\usepackage{imakeidx}
\usepackage[T1]{fontenc}
\usepackage{listings}

\makeindex
\titleformat{\chapter}[hang]{\normalfont\huge\bfseries}{\chaptertitlename\ \thechapter:}{1em}{} 

% Definició dels colors ---------------------------------------------
\definecolor{codegreen}{rgb}{0,0.6,0}
\definecolor{codegray}{rgb}{0.5,0.5,0.5}
\definecolor{codepurple}{rgb}{0.58,0,0.82}
\definecolor{backcolour}{rgb}{0.95,0.95,0.92}

% Definició de l'estil del codi 
\lstdefinestyle{mystyle}{
    backgroundcolor=\color{backcolour},   
    commentstyle=\color{codegreen},
    keywordstyle=\color{magenta},
    numberstyle=\tiny\color{codegray},
    stringstyle=\color{codepurple},
    basicstyle=\footnotesize,
    breakatwhitespace=false,         
    breaklines=true,                 
    captionpos=b,                    
    keepspaces=true,                 
    numbers=left,                    
    numbersep=5pt,                  
    showspaces=false,                
    showstringspaces=false,
    showtabs=false,                  
    tabsize=2
}
\lstset{style=mystyle}

\begin{document}

% Títol de la pràctica -----------------------------------------------
\begin{titlepage}
	\centering
	\includegraphics[width=0.55\textwidth]{udg_logo.png}\par\vspace{1cm}
	{\scshape\LARGE Universitat de girona \par}
	\vspace{1cm}
	{\scshape\Large Pràctica final\par}
	\vspace{1.5cm}
	{\huge\bfseries Fonaments de la computació\par}
	\vspace{2cm}
	{\Large\itshape Francesc Xavier Bullich Parra\par}
	{\Large\itshape Gil Gassó Rovira\par}
	{\Large\itshape Marc Sànchez Pifarré\par}
			
	\vfill
	Tutor de la pràctica\par
	Jaume Rigau

	\vfill

% Bottom of the page
	{\large \today\par}
\end{titlepage}

% Índex -----------------------------------------------
\tableofcontents
\clearpage


\chapter{Introduction}
\setcounter{chapter}{1}

\section{Definició del problema}

\begin{multicols}{2}

El problema presentat rep una entrada i accepta o rebutja en funció de l’entrada rebuda. L’entrada rebuda segons la definició, és un conjunt de llesques concatenades per el caràcter 13 del codi ASCII. 

En l’escenari presentat, una llesca es pot considerar un seguit de caràcters representats com una tira de símbols ‘+’ i ‘-’ on no importa l’ordre i que pot ser formada per un nombre de símbols >=0.

Donada una entrada com la descrita anteriorment, es tracta de trobar si una de les llesques que formen l’entrada conté una representació del patró definit a la \textcolor{red}{secció X.X}. Sí es pot afirmar l’existència del patró, l’entrada es dona com a accpetada, en cas contrari es dona com a rebutjada.
\end{multicols}

\section{Patró}

\begin{multicols}{2}

Per estudiar aquest problema tenim clar que només estem analitzant una llesca horitzontal donat un patró P on P és mínim. Es demana la cerca de 3 cel·les actives aïllades (considerant una cel·la activa aïllada quan està envoltada d’inactives) i flanquejades (ambdós costats) per un grup de cel·les actives seguides superior a 3.

\end{multicols}

Un exemple de llesca horitzontal que compleix el patró : 

\begin{lstlisting}[columns=fullflexible]
++++-+-+-+-++++ 
\end{lstlisting}

Un exemple de llesca horitzontal que no compleix el patró : 

\begin{lstlisting}[columns=fullflexible]
++++-+---+-++++ 
\end{lstlisting}

\clearpage

\textcolor{red}{Exemple de codi incrustat en latex}

\lstinputlisting[language=C++, caption=Matrix example]{Matrix.cpp}

\chapter{Finite Automata}

\chapter{Push-Down Automata}

\chapter{Turing Machine}
 
\printindex
\end{document}
